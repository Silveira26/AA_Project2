\section{Introduction}

The vertex cover problem, a cornerstone in graph theory, involves determining the existence of a \( k \)-vertex cover in an undirected graph \( G(V, E) \) with \( n \) vertices and \( m \) edges. Being NP-complete, this problem holds significance across domains like network design and optimization. This report investigates a randomized algorithm for solving the vertex cover problem. It provides a detailed examination of two distinct randomized approaches: the primary algorithm and an alternative method that includes a premature termination using a maximum number of attempts and the addition of not testing the same solution.  Through formal complexity analysis, empirical experiments (on expanding problem instances) and a comparative study of the results, this report aims to unveil the algorithm's strengths and practical implications.
The algorithms were developed in Python. The code is present in the \textbf{algorithms.py} script in the folder \textit{/code}. To run the program, you can access the root folder and run the batch file \textit{run.bat} or the following commands:

\begin{verbatim}
> cd .\code\
> python main.py
\end{verbatim}